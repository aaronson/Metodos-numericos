\section{Conclusión}

Este trabajo nos sirvió en primer lugar para entender cómo se puede modelar un problema físico de aplicación práctica con un sistema de ecuaciones lineales, a manera de ver una forma de aplicarlas en la vida real. El problema, que nos pedía calcular los valores de la isoterma $400º$ para un horno, presentaba la inherente dificultad de que no sabíamos la ecuación de la temperatura para cualquier punto dado del horno, sólo contábamos con las dimensiones del mismo y las temperaturas en las paredes interior y exterior, por lo que nos fue necesario inferir la función mediante una discretización de la misma en puntos de una matriz que eran estimables.

Una vez que la discretización estaba hecha, donde cada punto se estima como dependiendo de si mismo y de sus cuatro vecinos, se obtuvo una matriz esparsa que podía ser resuelta por cualquiera de los métodos vistos en clase.

Otro de los objetivos de este TP era resolver un sistema de ecuaciones lineales, y fue en este momento que hubo que llevarlo a la práctica. Para lograrlo, intentamos resolver la matriz con factorizacion $QR$, pero al resultar esto imposible, recurrimos al método de eliminación Gaussiana que es de más fácil implementación. Este método nos permitió triangular la matriz, que luego se pudo resolver con sustitución hacia atrás, para obtener los valores de temperatura en cada punto discretizado. Con estos valores, fue posible interpolar el valor del radio donde la temperatura es $400º$ para cada radio, al tomar el valor inmediatamente superior e inmediatamente inferior para cada uno.

Al observar los resultados obtenidos, fue evidente que una mayor cantidad de puntos en la discretización nos proporciona un resultado más exacto, aún cuando una discretización con pocos puntos nos proporcionaba un valor bastante aproximado. Con esto pudimos concluir que mientras más tienda a infinito la cantidad de puntos de la discretización, más y mas se va a acercar al valor de la función, una "discretización infinita" (que obviamente no seria una discretización) tendría el valor exacto de la función de la temperatura.

Otra de las cosas que podemos concluir es el orden del algoritmo, empíricamente obtuvimos que es al menos cuadrático, pero con pocos datos para analizar puesto que cuando crece mucho el tamaño de la discretización la máquina no puede calcularlo al tener memoria finita. Igualmente, viendo el algoritmo sabemos que éste tiene complejidad $O(n^3)$, por lo que es de esperar que si pudiéramos analizarlo en detalle, nuestro algoritmo mostraría ese comportamiento empíricamente.