\section{Introducción}

Si bien las computadoras nos permiten agilizar mucho el cálculo de operaciones aritméticas, siempre se debe tener en cuenta que poseen memoria finita. Por lo tanto, siempre que se quiera hacer una cuenta que involucre un operando infinito o una cantidad de operaciones infinitas, el resultado no va a poder ser calculado con exactitud y se va a tener que recurrir a una aproximación. Por supuesto, el avance tecnológico hace que esa aproximación sea cada vez más y más cercana al número exacto, pero aún así existe un error en esa aproximación.
La serie de Taylor es una forma conocida de calcular una función cuyo cálculo sea difícil (por ejemplo, una función trigonométrica) con operaciones más simples. Esta serie involucra una sumatoria de una cantidad de términos infinita, por lo tanto, si se quiere calcular una serie de Taylor con una computadora, se deberá acotar la cantidad de términos a usar, con el consiguiente error que trae aparejado. 
La función coseno es una función que tiene su conjunto imagen en los reales (por supuesto, acotado entre -1 y 1), por lo tanto, existen valores que no vamos a poder calcular con exactitud y debremos aproximar. Para este TP, vamos a analizar el comportamiento de la serie de Taylor alrededor del 0 (también llamada serie de McLaurin) para el coseno, la manera de aproximarla con una cantidad finita de términos, diferentes formas de calcularla y el error conllevado con cada una de ellas