\materia{M\'etodos Num\'ericos}
\submateria{Trabajo Pr\'actico 1}
\titulo{Errores}
\integrante{Aronson, Alex}{443/08}{alexaronson@gmail.com}
\integrante{Nahabedian, Leandro }{250/08}{leanahabedian@hotmail.com}
\integrante{Ravasi, Nicolás}{53/08}{nravasi@gmail.com}
\maketitle

\textbf{Resumen:} Se puede aproximar una función trigonométrica utilizando una sumatoria finita de términos, pero eso supone una aproximación. Además, los números con infinitos decimales necesitan ser truncados o redondeados por la máquina. Por lo tanto, intentar calcular el coseno de un número con una serie finita de McLaurin nos da una aproximación del resultado, buena o mala, pero aproximación al fin y al cabo. Analizaremos los errores involucrados, la propagación de los mismos, y las formas de aproximar y qué tan efectivas son a lo largo de este TP

\\
\\

\textbf{Palabras clave: McLaurin \ \ coseno \ \ aproximación \ \ mantisa} 
