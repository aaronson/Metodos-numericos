\section{Discusión}

Lo primero que salta a la vista es que la parametrización original dista mucho de ser la que realmente interpole los puntos de la curva que están a distancia $\Delta$, esto lo podemos observar visto que en ninguna de las tres parametrizaciones la parametrización por $\Delta * i$ nos da ni la cantidad ni los puntos que nos da la parametrización por longitud de arco.

Pero aún teniendo esto en cuenta, los $\Delta * i$ difieren notablemente, y esto se debe al tipo de parametrización que estamos usando, al usar una parametrización uniforme de 1 en 1 (Gráfico 1.1) y usar un $\Delta = 0.5$, los $t_\Delta$ quedan a exactamente $t_{\frac{i}{2}}$ de distancia, o sea, no se está reparametrizando nada, simplemente se está recorriendo la curva en función de la distancia de los puntos entre sí, más rápidamente cuando dos puntos sucesivos están más separados (por lo que solo va a haber $2 *\Delta$ de distancia entre ellos). Por lo tanto, parametrizar de esta manera no otorga nueva información.

En la parametrización chord-length (Gráfico 1.2), al ser la distancia entre dos samples no constante, la parametrización es un tanto mejor, pero aún así podemos ver que existen diferencias en los vectores tangentes, por lo tanto, existen secciones donde la curva se recorre a mayor velocidad y otras a menor.

La centrípeta (Gráfico 1.3), como la chord-length, toma en cuenta la distancia al parametrizar, por lo tanto, no es independiente de la curva como la uniforme, pero a su vez, al ser la raíz cuadrada de la distancia, esta magnitud no tiene tanto peso, es por eso que observamos puntos más o menos separados de acuerdo a la sección de la curva.

En las reparametrizaciones, en todos los casos podemos observar que se hace un recorrido más efectivo de la curva. Esto quiere decir, los puntos parecen estar a distancias sobre la curva constantes, y los vectores tangentes aparentan tener valores idénticos. A su vez, no se aprecia una  diferencia notable entre los tipos de parametrización con respecto a las distancias y tangentes, esto es, el tipo de parametrización no afecta en gran manera a la curva si luego se la reparametriza. A modo ilustrativo de lo que queremos decir, miramos la cantidad de puntos que tenía la reparametrización en cada caso: 90 la uniforme, 89 la chord-length y 83 la centrípeta. Esto nos hace pensar que al reparametrizar la curva, uno se está "independizando" de la parametrización original.

Pero obviamente sigue existiendo el hecho de que la parametrización original conlleva splines distintas, esto lo podemos ver en el hecho de que las curvas no describen las mismas ondulaciones, por lo tanto, al reparametrizar, se está independizando de la parametrización en la manera de recorrer la curva, pero no así de la curva misma, porque esta depende únicamente de los coeficientes que se hayan usado, y estos dependen a su vez de la parametrización original

En una sucesión de puntos al azar, podemos ver que las reparametrizaciones describen curvas similares, con algunas diferencias, pero la forma es parecida. Ahora bien, usando un caso hecho especialmente por nosotros, con la intención de hacer notorias las diferencias, estas aparecen contrastando la uniforme con el resto. Al usar un set de puntos especiales donde se alternan puntos muy cercanos entre sí con puntos muy alejados entre sí, podemos ver que la trayectoria de la curva uniforme (Gráfico 3.1) difiere notablemente de la chord-length (Gráfico 3.2), esto se debe a que la uniforme asigna los $t$ independientemente de la distancia entre puntos sucesivos,  por lo tanto, no toma en cuenta si la distancia es grande o chica, el chord-length sí, por lo tanto, la parametrización diferirá notablemente.

Al variar el $\Delta$, observamos, aparte de lo esperado que los puntos $t_\Delta$ se alejen, los vectores tangentes no son constantes, esto podemos atribuirlo a un mayor error, porque estamos integrando sobre un intervalo más grande, es más probable cometer errores dado que estamos aproximando la integración.


