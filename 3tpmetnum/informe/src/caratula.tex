\materia{M\'etodos Num\'ericos}
\submateria{Trabajo Pr\'actico 3}
\titulo{Interpolación}
\integrante{Aronson, Alex}{443/08}{alexaronson@gmail.com}
\integrante{Nahabedian, Leandro }{250/08}{leanahabedian@hotmail.com}
\integrante{Ravasi, Nicolás}{53/08}{nravasi@gmail.com}
\maketitle

\textbf{Resumen:}
En este informe explicamos los métodos numéricos necesarios para poder obtener nuevos puntos partiendo de un conjunto discreto de puntos. Para esto utilizamos la interpolación parametrica ya que los puntos que obteniamos estaban en dos dimensiones.
Mas luego, pasamos a utilizar una técnica algoritmica para obtener los puntos por sobre la curva del spline que cumplan que la distancia entre cualquier par de puntos adyacentes sean iguales entre si.

El resto de nuestro trabajo se enfoca en realizar comparaciones entre los distintos métodos de parametrizar la curva, la cual nos otorgo algunos resultados interesantes que detallaremos mas adelante.

\\\\\\
\\

\textbf{Palabras clave: Spline -  Parametrización  - Busqueda Binaria  - Integración  } 
